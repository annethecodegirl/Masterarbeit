\documentclass[12pt,a4paper]{article}

\usepackage[utf8]{inputenc}
\usepackage[T1]{fontenc}
\usepackage{lmodern}
\usepackage{amsmath, amssymb}
\usepackage{graphicx}
\usepackage{caption}
\usepackage{subcaption}
\usepackage{booktabs}
\usepackage{geometry}
\usepackage{natbib}
\usepackage{setspace}
\usepackage{hyperref}

\geometry{left=3cm, right=2cm, top=2cm, bottom=2cm}
\onehalfspacing

\begin{document}


\subsection{Life Cycle Assessment (LCA) Method}
The LCA method calculates emissions throughout the entire life cycle of a product or service, from production to disposal. This model captures emissions from every stage of the supply chain and provides a comprehensive assessment of indirect emissions.

The carbon footprint for a single industry using the LCA approach is:
\[
fp_h = q_h \cdot \text{LCA}_j
\]
where \(q_h\) is the quantity consumed by household \(h\), and \(\text{LCA}_j\) represents the life cycle emissions per unit in industry \(j\).


\subsubsection{Methodology for Household Carbon Footprint Calculation based on LCA approach}

The methodology developed by Peng et al. (2021) provides a comprehensive framework for calculating household carbon footprints by integrating life-cycle assessment (LCA) approaches. This framework accounts for both carbon emissions and sequestration from various household activities, including consumption and production, using survey data. It employs three primary LCA methods: (1) \textit{Process LCA}, which evaluates emissions from agricultural and livestock-related processes, capturing material inputs like fertilizers and operational activities; (2) \textit{Input–Output LCA}, applied to household consumption activities such as energy, food, housing, and transportation; and (3) \textit{Hybrid LCA}, which combines process and input-output methods to assess afforestation activities and durable goods like clothing. The methodology categorizes household activities into specific domains, including direct energy consumption, living consumption (short-lived and durable goods), agricultural activities (emissions from material inputs and sequestration from biomass growth), afforestation (carbon sequestration from tree plantations such as citrus farming), and livestock raising (emissions from fodder preparation, livestock growth, and manure management). The total carbon footprint is expressed as the sum of emissions and sequestration across these domains, incorporating emission factors and material inputs derived from IPCC guidelines and regional data. 
\subsubsection*{Overall Carbon Footprint}
\begin{equation}
CF_i = \sum_{n} E_{in} + \sum_{m} S_{im}
\end{equation}
where $CF_i$ represents the Carbon footprint of household $i$, $E_{in}$ is the annual carbon emissions of household $i$ in category $n$ and $S_{im}$ is the annual carbon sequestration of household $i$ in category $m$.


\subsubsection*{Carbon Emissions from Direct Energy Consumption}
\begin{equation}
E_{id} = \sum_d (F_{id} \cdot EF_d)
\end{equation}
\begin{equation}
EF_d = OX_d \cdot \left(C_{o,d} \cdot \frac{12}{44} + C_{h,d} \cdot \frac{12}{16}\right) \cdot H_d \cdot 10^{-9}
\end{equation}

where $E_{id}$ is the carbon emissions from direct fuel consumption, $F_{id}$ is the fuel consumption of household $i$ for fuel type $d$, $EF_d$ is the emission factor of fuel $d$, $OX_d$ is the oxygenation efficiency (assumed 100\%), $C_{o,d}$ and $C_{h,d}$ are the CO$_2$ and CH$_4$ emission factors, and $H_d$ is the net calorific value of the fuel.


\subsubsection*{Carbon Emissions from Living Consumption}
\begin{equation}
E_{if} = \sum_f (EF_f \cdot C_{if})
\end{equation}
\begin{equation}
E_{ij} = \sum_j \frac{(EF_j \cdot C_{ij})}{L_j}
\end{equation}
where: $E_{if}$ and $E_{ij}$ are the carbon emissions from short-lived and durable consumer products, $C_{if}$ and $C_{ij}$ are the amounts of consumed material, and $L_j$ is the lifetime of durable consumer product $j$.


\subsubsection*{Carbon Footprint in Agricultural Activities}
\begin{equation}
CF_{ia} = \sum_a (EF_a \cdot M_{ia}) + \sum_t (EF_t \cdot FS_{ia}) + \sum_v (B_v \cdot 0.475)
\end{equation}
where: $CF_{ia}$ is the carbon footprint from agricultural activities, $EF_a$ and $EF_t$ are the emission factors for materials and field operations, $M_{ia}$ is the material input, $FS_{ia}$ is the field size, and $B_v$ is the biomass produced.


\subsubsection*{Carbon Sequestration from Afforestation}
\begin{equation}
S_{iaf} = FS_{iaf} \cdot CS_{\text{citrus}}
\end{equation}
where: $S_{iaf}$ is the carbon sequestration from afforestation, $FS_{iaf}$ is the field size for afforestation, and $CS_{\text{citrus}}$ is the carbon stock of citrus trees.


\subsubsection*{Carbon Emissions from Livestock Raising}
\begin{equation}
E_{il} = \sum_f (EF_{if} \cdot F_{if}) + \sum_l (EF_{il} \cdot N_{il})
\end{equation}
where: $E_{il}$ is the carbon emissions from livestock raising, $EF_{if}$ and $EF_{il}$ are the emission factors for fodder and livestock, $F_{if}$ is the fodder consumption, and $N_{il}$ is the number of livestock.



\subsubsection*{Aggregate Formula for Household Carbon Footprint}

The total carbon footprint ($CF_{\text{total}}$) of a household is the sum of emissions and sequestration from all relevant activities, including direct energy consumption, living consumption, agricultural activities, afforestation, and livestock raising:

\begin{align}
CF_{\text{total}} = & \underbrace{\sum_d \left(F_{id} \cdot EF_d\right)}_{\text{Direct energy consumption}} + 
\underbrace{\sum_f \left(EF_f \cdot C_{if}\right) + \sum_j \frac{\left(EF_j \cdot C_{ij}\right)}{L_j}}_{\text{Living consumption}} \nonumber \\
& + \underbrace{\sum_a \left(EF_a \cdot M_{ia}\right) + \sum_t \left(EF_t \cdot FS_{ia}\right) + \sum_v \left(B_v \cdot 0.475\right)}_{\text{Agricultural activities}} \nonumber \\
& - \underbrace{\sum_{iaf} \left(FS_{iaf} \cdot CS_{\text{citrus}}\right)}_{\text{Afforestation}} +
\underbrace{\sum_f \left(EF_{if} \cdot F_{if}\right) + \sum_l \left(EF_{il} \cdot N_{il}\right)}_{\text{Livestock raising}}
\end{align}


The total household carbon footprint is denoted by $CF_{\text{total}}$. Fuel consumption for fuel type $d$ is represented by $F_{id}$, and the emission factor of fuel $d$ is denoted by $EF_d$. The amount of consumed materials for short-lived ($f$) and durable products ($j$) is represented by $C_{if}$ and $C_{ij}$, respectively, with the lifetime of durable product $j$ given by $L_j$. The emission factors for short-lived and durable products are represented by $EF_f$ and $EF_j$, respectively. The material input for agricultural activity $a$ is denoted by $M_{ia}$, while the emission factors for agricultural materials and field operations are given by $EF_a$ and $EF_t$. The field size for agricultural activities is represented by $FS_{ia}$, and the biomass produced is denoted by $B_v$. The field size for afforestation is represented by $FS_{iaf}$, while the carbon stock of citrus trees is given by $CS_{\text{citrus}}$. Fodder consumption for livestock is denoted by $F_{if}$, and the number of livestock is represented by $N_{il}$. The emission factors for fodder and livestock are represented by $EF_{if}$ and $EF_{il}$, respectively.
\end{document}


