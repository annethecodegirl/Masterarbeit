\documentclass{article}
\usepackage[utf8]{inputenc}
\usepackage{booktabs}
\usepackage{caption}
\usepackage{amsmath}

\begin{document}
\section{Empirical Illustration: Application of the Single-Industry Model}

Here, the simplified version of the Hakenes and Schliephake (2024) model is applied to the U.S. wheat market, using USDA data from 2010 to 2016. Production volumes serve as a proxy for quantity supplied, while total domestic use approximates quantity demanded. Farm prices are taken as observed average annual prices.

To estimate supply behavior, an ordinary least squares (OLS) regression of price is fitted on observed production, yielding the empirical supply curve. In the empirical illustration, the demand curve is specified as linear and downward sloping. Its slope is calibrated using average values from the dataset, consistent with observed market behavior in the U.S. wheat sector. While the curve is not estimated directly via regression (due to data limitations on price responsiveness), it reflects a stylized elasticity based on domain knowledge. This contrasts with the supply curve, which is estimated using OLS on observed price and production data. We simulate the 2016–2017 wheat supply shock, during which production declined by 15.6\%. The intersection of the two curves provide the empirical equilibrium quantities and prices before and after the 2016–2017 supply shock. This is modeled by proportionally shifting the supply curve upward. Equilibrium price and quantity before and after the shock are obtained by solving the intersection between the demand curve and the respective supply curves.


\subsection*{Carbon Footprint Calculation from Data}

We compute the carbon footprint associated with each equilibrium using an emission factor of 10.88 kg CO\textsubscript{2}e per bushel (based on FAO and USDA estimates).

\begin{table}[ht]
\centering
\begin{tabular}{lccc}
\toprule
\textbf{Scenario} & \textbf{Quantity} & \textbf{Price} & \textbf{Carbon Footprint} \\
\textbf{USDA data} & \textbf{(million bushels)} & \textbf{(USD)} & \textbf{(million kg CO\textsubscript{2}e)} \\
\midrule
Before Shock  & 2100.71 & 5.58 & 22859.68 \\
After Shock & 2068.38 & 5.82 & 22500.32 \\
\midrule
\textbf{Change} & \textemdash & \textemdash & \textbf{-359.36} \\
\bottomrule
\end{tabular}
\caption{Carbon footprint before and after the supply shock using real market data.}
\end{table}

\subsection*{Theoretical Model-Based Calculation}

To simulate the same supply shock within the Hakenes and Schliephake (2024) framework, the demand curve from the empirical estimation was retained. However, instead of using a supply curve estimated via ordinary least squares, a theoretically derived supply curve was constructed based on model assumptions. In this approach, firms were assumed to raise capital from households, who in turn optimally allocate their investments under risk.

The equilibrium supply curve in this setup is derived from the market-clearing condition and the household's optimal investment response under uncertainty, and takes the form:
\[
P(Q) = c(r_f - \lambda) + \frac{c^2 \alpha \sigma^2}{n - 1} Q,
\]
where \( c \) denotes the marginal cost of production, \( r_f \) the risk-free rate, \( \lambda \) the liquidation value of capital, \( \alpha \) the coefficient of absolute risk aversion, \( \sigma^2 \) the variance of investment returns, and \( n \) the total number of households. 

This expression yields a linear and upward-sloping supply curve. The theoretical supply curve applied in this illustration was constructed using parameter values selected to reflect realistic conditions in the U.S. wheat and financial markets during the study period. The marginal cost of production was assumed to be $c = 4$, which is consistent with per-bushel production costs observed in U.S. wheat farming and allows the resulting equilibrium prices to align with historical market levels. The risk-free rate was set to $r_f = 0.05$, corresponding to the average yield on 10-year U.S. Treasury bonds between 2010 and 2016. The liquidation value of capital was taken as $\lambda = 0.01$, reflecting the reduced resale value of farm-specific capital such as machinery or equipment. The coefficient of absolute risk aversion was assumed to be $\alpha = 0.5$, a value that captures moderate household risk sensitivity consistent with empirical estimates from investment literature. The volatility of investment returns was specified as $\sigma = 0.4$, implying a variance of $\sigma^2 = 0.16$, which falls within the range typically observed for U.S. agricultural investments and related financial instruments. Finally, the number of households was assumed to be $n = 100{,}000$, representing an approximation of the number of wheat-producing farms in the United States during the relevant years. These parameter values were used to generate a supply curve that reflects theoretical investment behavior under risk, providing a basis for comparison with the empirically estimated curve, and the intercept reflects the opportunity cost of capital. The new equilibrium values were obtained by solving the intersection of this supply curve with the demand curve used previously.

By solving the intersection of this supply curve with the same demand curve used in the empirical case, equilibrium values for price and quantity were obtained both before and after the simulated shock. The corresponding carbon footprints were then computed using the same emissions factor of 10.88 kg CO\textsubscript{2}e per bushel.

\begin{table}[ht]
\centering
\begin{tabular}{lccc}
\toprule
\textbf{Scenario} & \textbf{Quantity} & \textbf{Price} & \textbf{Carbon Footprint} \\
\textbf{Theory} & \textbf{(million bushels)} & \textbf{(USD)} & \textbf{(million kg CO\textsubscript{2}e)} \\
\midrule
Before Shock & 2112.36 & 5.69 & 22983.46 \\
After Shock   & 2095.13 & 5.85 & 22808.35 \\
\midrule
\textbf{Change} & \textemdash & \textemdash & \textbf{-175.11} \\
\bottomrule
\end{tabular}
\caption{Model-based carbon footprint before and after the supply shock.}
\end{table}

\subsection*{Discussion: Interpreting the Difference in Results}

Although the same demand curve was used in both the empirical and theoretical approaches, the estimated reduction in carbon footprint differed considerably. The empirical estimation yielded a reduction of 359.36 million kg CO\textsubscript{2}e, while the theoretical model predicted a more modest reduction of 175.11 million kg CO\textsubscript{2}e.

This difference can be attributed entirely to the way supply was modeled. In the empirical estimation, the supply curve was estimated via OLS using observed data on price and quantity. This approach captured market behavior as it appeared in the historical record but did not account for underlying decision-making under uncertainty or equilibrium responses. In contrast, the theoretical supply curve was derived from the model’s structural assumptions, incorporating risk preferences, investment volatility, and optimal capital allocation. It reflected how households would respond to market changes under forward-looking behavior, leading to a more muted response in output and, correspondingly, in emissions.

Additionally, the theoretical model introduced a consequentialist perspective by assigning carbon responsibility based on the marginal impact of a household's consumption or investment. In doing so, it internalized substitution effects and capital reallocation, which were not accounted for in the empirical estimation. As a result, while the same emissions formula was applied in both cases, the theoretical model predicted a smaller footprint change due to the buffering effects of equilibrium adjustments. This difference underscores the importance of integrating behavioral dynamics into footprint assessment, particularly when evaluating the impact of shocks or policy interventions.

\end{document}
