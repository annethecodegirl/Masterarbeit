\documentclass[12pt,a4paper]{article}

\usepackage[utf8]{inputenc}
\usepackage[T1]{fontenc}
\usepackage{lmodern}
\usepackage{amsmath, amssymb}
\usepackage{geometry}
\usepackage{natbib}
\usepackage{setspace}
\usepackage{hyperref}
\usepackage{booktabs}

\geometry{left=3cm, right=2cm, top=2cm, bottom=2cm}
\onehalfspacing

\begin{document}

\section*{The Hakenes \& Schliephake Model}

Traditional methods for estimating household carbon footprints attribute emissions based on direct consumption or financial ownership in emitting industries. However, they often ignore the market feedback loops triggered by individual decisions — such as how a household reducing demand might simply shift that demand to other consumers or investors.

The model developed by Hakenes and Schliephake (2024) addresses this issue through a general equilibrium framework. By embedding both product and financial markets, the model assigns carbon footprints based not only on what households consume or invest in, but also on the spillover effects of those choices across the economy. This consequentialist approach attempts to capture the true marginal impact of household behavior on aggregate emissions.

\section{Deriving the Household Footprint in a One-Industry Economy}

We consider a simplified version of the model developed by Hakenes and Schliephake (2024), focusing on an economy with a single industry. A representative good is produced using capital as the only input. Firms operate under constant returns to scale, with a marginal cost of production $c$. Let $Q$ denote the aggregate quantity produced and consumed, and $I$ the total capital invested. Given the linear technology, we have:

\[
I = cQ.
\]

Each unit of the good generates emissions $x$, which aggregates both production-related and consumption-related emissions. Thus, total emissions in the economy are given by:

\[
X = xQ.
\]

\subsection*{Firms and Capital Market}

Firms raise capital $I$ from households and produce output $Q$. After selling the output at price $P$, they repay investors using the liquidation value $\lambda$ and a noise term $\varepsilon$, which follows a normal distribution with zero mean and variance $\sigma^2$. The return on investment is:

\[
r = \frac{P}{c} + \lambda + \varepsilon.
\]

Profits are distributed to investors in proportion to their capital contributions. Firms operate competitively, so expected profits are zero in equilibrium.

\subsection*{Household Optimization Problem}

Household $h$ is endowed with wealth $w$ and allocates it between investment $i_h$ and consumption $q_h$. The portion not invested yields a risk-free return $r_f$. The budget constraint is:

\[
m_h = r i_h + r_f(w - i_h) - P q_h,
\]

where $m_h$ is the leftover wealth after investment and consumption. The household derives utility from consumption and terminal wealth. The expected utility function is given by:

\[
U_h = \mathbb{E} \left[ -e^{-\alpha \left( a q_h - \frac{b}{2}q_h^2 + m_h - xQ \right)} \right],
\]

where $a$ represents the marginal utility of the first unit of the good, $b > 0$ captures diminishing marginal utility, $\alpha$ is the coefficient of absolute risk aversion, and $xQ$ reflects the disutility from global emissions.

Substituting $m_h$ into the utility function and linearizing expectations due to the exponential-normal structure, we obtain:

\[
\mathbb{E}[U_h] = -\exp \left\{ -\alpha \left[ (a - P)q_h - \frac{b}{2}q_h^2 + r_f w + \left( \frac{P}{c} + \lambda - r_f \right)i_h - \frac{\alpha}{2} \sigma^2 i_h^2 - xQ \right] \right\}.
\]

\subsection*{Market Equilibrium and Footprint Derivation}

To calculate the household's consequentialist footprint, we compare the equilibrium outcome with and without household $h$. In equilibrium, the market clears:

\[
Q = q_h + (n - 1) q_{-h}, \quad I = i_h + (n - 1) i_{-h}, \quad I = cQ.
\]

Other households maximize the same utility, taking $P$ as given. Their optimal demand and investment are derived from the first-order conditions:

\[
q_{-h} = \frac{a - x - P}{b}, \quad i_{-h} = \frac{1}{\alpha \sigma^2} \left( \frac{P}{c} + \lambda - r_f \right).
\]

Substituting these into the equilibrium conditions and solving, we obtain the aggregate quantity:

\[
Q = \phi q_h + (1 - \phi) \frac{i_h}{c} + \text{(terms independent of } h),
\]

where the weighting parameter $\phi$ is defined as:

\[
\phi = \frac{b}{b + c^2 \alpha \sigma^2}.
\]

This weight determines how the household’s choices affect equilibrium quantities and, consequently, emissions. The consequentialist footprint of household $h$ is defined as the marginal impact of their participation on total emissions:

\[
fp_h = x \left( Q(q_h, i_h) - Q(0, 0) \right) = x \left( \phi q_h + (1 - \phi) \frac{i_h}{c} \right).
\]


The parameter $\phi$ captures the relative influence of consumption and investment. When the financial asset is risk-free ($\sigma^2 = 0$), we obtain $\phi = 1$, and the entire footprint is attributed to consumption. Conversely, if consumption utility is linear ($b = 0$), then $\phi = 0$, and the footprint depends entirely on investment. This formulation ensures full accounting of emissions across households:

\[
\sum_h fp_h = xQ = X.
\]
\section{Derivation of the Weighting Parameter \( \boldsymbol{\phi} \)}

To derive the footprint weighting parameter \( \boldsymbol{\phi} \), we begin with the assumption that aggregate output \( Q \) is produced by a linear technology using capital \( I \) with constant marginal cost \( c \). Hence,
\[
Q = \frac{I}{c}.
\]

The total capital in the market is supplied by \( n \) households. We distinguish a representative household \( h \) from the remaining \( n - 1 \) households, and denote their investment and consumption decisions by \( (i_h, q_h) \) and \( (i_{-h}, q_{-h}) \), respectively.

In equilibrium, market clearing implies:
\[
Q = q_h + (n - 1) q_{-h}, \quad I = i_h + (n - 1) i_{-h}, \quad I = cQ.
\]

Substituting into the identity \( I = cQ \), we obtain:
\[
i_h + (n - 1) i_{-h} = c \left( q_h + (n - 1) q_{-h} \right).
\]

Now consider how the quantity \( Q \) changes when household \( h \) changes its behavior. Holding the other households' behavior fixed, the marginal effect of \( h \)'s consumption and investment on output is given by the total differential:
\[
\frac{\partial Q}{\partial q_h} = 1, \quad \frac{\partial Q}{\partial i_h} = \frac{1}{c}.
\]

However, these effects are attenuated by the endogenous reactions of other households. If household \( h \) increases consumption \( q_h \), market price \( P \) rises. Other households respond by lowering their own consumption \( q_{-h} \) and adjusting their investment \( i_{-h} \) to the new return. Conversely, if \( h \) increases investment \( i_h \), the capital supply rises, which reduces price and affects others' choices.

We now derive the explicit behavioral responses.

The other households' optimal consumption satisfies:
\[
\frac{\partial \mathbb{E}[U_{-h}]}{\partial q_{-h}} = 0 \quad \Rightarrow \quad a - x - b q_{-h} - P = 0,
\]
which yields:
\[
q_{-h} = \frac{a - x - P}{b}.
\]

Their optimal investment satisfies:
\[
\frac{\partial \mathbb{E}[U_{-h}]}{\partial i_{-h}} = 0 \quad \Rightarrow \quad \frac{P}{c} + \lambda - r_f - \alpha \sigma^2 i_{-h} = 0,
\]
so that:
\[
i_{-h} = \frac{1}{\alpha \sigma^2} \left( \frac{P}{c} + \lambda - r_f \right).
\]

Now insert these behavioral responses into the aggregate equilibrium conditions:
\[
Q = q_h + (n - 1) \left( \frac{a - x - P}{b} \right), \quad I = i_h + (n - 1) \left( \frac{1}{\alpha \sigma^2} \left( \frac{P}{c} + \lambda - r_f \right) \right).
\]

Combining these with \( Q = \frac{I}{c} \), we solve for the dependence of \( Q \) on \( q_h \) and \( i_h \). Define the partial footprint of household \( h \) as the difference in total output caused by its activity:
\[
fp_h = x \left( Q(q_h, i_h) - Q(0, 0) \right).
\]

Linearizing \( Q \) in \( q_h \) and \( i_h \), and denoting the resulting coefficients as footprint weights, we obtain:
\[
fp_h = x \left( \phi q_h + (1 - \phi) \frac{i_h}{c} \right),
\]
where
\[
\phi = \frac{b}{b + c^2 \alpha \sigma^2}.
\]

This expression reflects how much of the household’s carbon footprint is attributed to consumption versus investment. It arises from the equilibrium interactions between price responses and household behavioral elasticities in both the product and capital markets.

\section{Comparative Statics of the Weighting Parameter \( \boldsymbol{\phi} \)}

We now investigate how the footprint weighting parameter \( \boldsymbol{\phi} \), defined as
\[
\boldsymbol{\phi} = \frac{b}{b + c^2 \alpha \sigma^2},
\]
responds to changes in the underlying structural parameters of the model.

Differentiating \( \boldsymbol{\phi} \) with respect to the coefficient of absolute risk aversion \( \alpha \), we obtain
\[
\frac{\partial \boldsymbol{\phi}}{\partial \alpha} = -\frac{b c^2 \sigma^2}{(b + c^2 \alpha \sigma^2)^2} < 0.
\]
This implies that as households become more risk-averse, the footprint share attributed to consumption declines, while the relative importance of investment decisions increases.

With respect to the volatility of financial returns, captured by \( \sigma^2 \), we find
\[
\frac{\partial \boldsymbol{\phi}}{\partial \sigma^2} = -\frac{b c^2 \alpha}{(b + c^2 \alpha \sigma^2)^2} < 0.
\]
An increase in financial risk similarly reduces \( \boldsymbol{\phi} \), shifting the footprint burden from consumption to investment channels.

Finally, consider the effect of changing the curvature of the utility function through the parameter \( b \). Differentiation yields
\[
\frac{\partial \boldsymbol{\phi}}{\partial b} = \frac{c^2 \alpha \sigma^2}{(b + c^2 \alpha \sigma^2)^2} > 0.
\]
A higher value of \( b \), indicating stronger diminishing marginal utility from consumption, increases the share of the footprint attributed to consumption activities.

In sum, the weighting parameter \( \boldsymbol{\phi} \) is decreasing in both risk aversion and return volatility, and increasing in the concavity of consumption preferences. These results highlight how the relative responsibility of consumption and investment for carbon emissions is endogenous to household behavior and financial risk, making the model responsive to empirical variation across households or economies.

\section{Empirical Illustration: Application of the Single-Industry Model}

Here, the simplified version of the Hakenes and Schliephake (2024) model is applied to the U.S. wheat market, using USDA data from 2010 to 2016. Production volumes serve as a proxy for quantity supplied, while total domestic use approximates quantity demanded. Farm prices are taken as observed average annual prices.

To estimate supply behavior, an ordinary least squares (OLS) regression of price is fitted on observed production, yielding the empirical supply curve. In the empirical illustration, the demand curve is specified as linear and downward sloping. Its slope is calibrated using average values from the dataset, consistent with observed market behavior in the U.S. wheat sector. While the curve is not estimated directly via regression (due to data limitations on price responsiveness), it reflects a stylized elasticity based on domain knowledge. This contrasts with the supply curve, which is estimated using OLS on observed price and production data. We simulate the 2016–2017 wheat supply shock, during which production declined by 15.6\%. The intersection of the two curves provide the empirical equilibrium quantities and prices before and after the 2016–2017 supply shock. This is modeled by proportionally shifting the supply curve upward. Equilibrium price and quantity before and after the shock are obtained by solving the intersection between the demand curve and the respective supply curves.


\subsection*{Carbon Footprint Estimation under Empirical Supply Curve}

We compute the carbon footprint associated with each equilibrium using an emission factor of 10.88 kg CO\textsubscript{2}e per bushel (based on FAO and USDA estimates).

\begin{table}[ht]
\centering
\begin{tabular}{lccc}
\toprule
\textbf{Scenario} & \textbf{Quantity} & \textbf{Price} & \textbf{Carbon Footprint} \\
\textbf{USDA data} & \textbf{(million bushels)} & \textbf{(USD)} & \textbf{(million kg CO\textsubscript{2}e)} \\
\midrule
Before Shock  & 2100.71 & 5.58 & 22859.68 \\
After Shock & 2068.38 & 5.82 & 22500.32 \\
\midrule
\textbf{Change} & \textemdash & \textemdash & \textbf{-359.36} \\
\bottomrule
\end{tabular}
\caption{Carbon footprint before and after the supply shock using real market data.}
\end{table}

\subsection*{Carbon Footprint Estimation under Theoretical Supply Curve}

To simulate the same supply shock within the Hakenes and Schliephake (2024) framework, the demand curve from the empirical estimation was retained. However, instead of using a supply curve estimated via ordinary least squares, a theoretically derived supply curve was constructed based on model assumptions. In this approach, firms were assumed to raise capital from households, who in turn optimally allocate their investments under risk.

The equilibrium supply curve in this setup is derived from the market-clearing condition and the household's optimal investment response under uncertainty, and takes the form:
\[
P(Q) = c(r_f - \lambda) + \frac{c^2 \alpha \sigma^2}{n - 1} Q,
\]
where \( c \) denotes the marginal cost of production, \( r_f \) the risk-free rate, \( \lambda \) the liquidation value of capital, \( \alpha \) the coefficient of absolute risk aversion, \( \sigma^2 \) the variance of investment returns, and \( n \) the total number of households. 

This expression yields a linear and upward-sloping supply curve. The theoretical supply curve applied in this illustration was constructed using parameter values selected to reflect realistic conditions in the U.S. wheat and financial markets during the study period. The marginal cost of production was assumed to be $c = 4$, which is consistent with per-bushel production costs observed in U.S. wheat farming and allows the resulting equilibrium prices to align with historical market levels. The risk-free rate was set to $r_f = 0.05$, corresponding to the average yield on 10-year U.S. Treasury bonds between 2010 and 2016. The liquidation value of capital was taken as $\lambda = 0.01$, reflecting the reduced resale value of farm-specific capital such as machinery or equipment. The coefficient of absolute risk aversion was assumed to be $\alpha = 0.5$, a value that captures moderate household risk sensitivity consistent with empirical estimates from investment literature. The volatility of investment returns was specified as $\sigma = 0.4$, implying a variance of $\sigma^2 = 0.16$, which falls within the range typically observed for U.S. agricultural investments and related financial instruments. Finally, the number of households was assumed to be $n = 100{,}000$, representing an approximation of the number of wheat-producing farms in the United States during the relevant years. These parameter values were used to generate a supply curve that reflects theoretical investment behavior under risk, providing a basis for comparison with the empirically estimated curve, and the intercept reflects the opportunity cost of capital. The new equilibrium values were obtained by solving the intersection of this supply curve with the demand curve used previously.

By solving the intersection of this supply curve with the same demand curve used in the empirical case, equilibrium values for price and quantity were obtained both before and after the simulated shock. The corresponding carbon footprints were then computed using the same emissions factor of 10.88 kg CO\textsubscript{2}e per bushel.

\begin{table}[ht]
\centering
\begin{tabular}{lccc}
\toprule
\textbf{Scenario} & \textbf{Quantity} & \textbf{Price} & \textbf{Carbon Footprint} \\
\textbf{Theory} & \textbf{(million bushels)} & \textbf{(USD)} & \textbf{(million kg CO\textsubscript{2}e)} \\
\midrule
Before Shock & 2112.36 & 5.69 & 22983.46 \\
After Shock   & 2095.13 & 5.85 & 22808.35 \\
\midrule
\textbf{Change} & \textemdash & \textemdash & \textbf{-175.11} \\
\bottomrule
\end{tabular}
\caption{Model-based carbon footprint before and after the supply shock.}
\end{table}

\subsection*{Structural Sources of Difference in Emissions Outcomes}

Although the same demand curve was used in both the empirical and theoretical approaches, the estimated reduction in carbon footprint differed considerably. The empirical estimation yielded a reduction of 359.36 million kg CO\textsubscript{2}e, while the theoretical model predicted a more modest reduction of 175.11 million kg CO\textsubscript{2}e.

This difference can be attributed entirely to the way supply was modeled. In the empirical estimation, the supply curve was estimated via OLS using observed data on price and quantity. This approach captured market behavior as it appeared in the historical record but did not account for underlying decision-making under uncertainty or equilibrium responses. In contrast, the theoretical supply curve was derived from the model’s structural assumptions, incorporating risk preferences, investment volatility, and optimal capital allocation. It reflected how households would respond to market changes under forward-looking behavior, leading to a more muted response in output and, correspondingly, in emissions.

Additionally, the theoretical model introduced a consequentialist perspective by assigning carbon responsibility based on the marginal impact of a household's consumption or investment. In doing so, it internalized substitution effects and capital reallocation, which were not accounted for in the empirical estimation. As a result, while the same emissions formula was applied in both cases, the theoretical model predicted a smaller footprint change due to the buffering effects of equilibrium adjustments. This difference underscores the importance of integrating behavioral dynamics into footprint assessment, particularly when evaluating the impact of shocks or policy interventions.


\end{document}
